\documentclass[11pt]{article}

\usepackage{natbib}
\usepackage{setspace}
\usepackage[left=2.5cm,top=2.8cm,right=2.5cm,bottom=2.8cm]{geometry}
\usepackage{graphicx}
\usepackage{amsmath}
\usepackage{theorem}
\usepackage{version}
\usepackage{multirow}
\usepackage{amssymb}
\usepackage{tikz}
\usetikzlibrary{arrows,arrows.meta,decorations,decorations.pathreplacing,calc,matrix}

\definecolor{Red}{rgb}{1,0,0}
\definecolor{Blue}{rgb}{0,0,1}
\definecolor{Green}{rgb}{0,1,0}
\definecolor{magenta}{rgb}{1,0,.6}
\definecolor{lightblue}{rgb}{0,.5,1}
\definecolor{lightpurple}{rgb}{.6,.4,1}
\definecolor{gold}{rgb}{.6,.5,0}
\definecolor{orange}{rgb}{1,0.4,0}
\definecolor{hotpink}{rgb}{1,0,0.5}
\definecolor{newcolor2}{rgb}{.5,.3,.5}
\definecolor{newcolor}{rgb}{0,.3,1}
\definecolor{newcolor3}{rgb}{1,0,.35}
\definecolor{darkgreen1}{rgb}{0, .35, 0}
\definecolor{darkgreen}{rgb}{0, .6, 0}
\definecolor{darkred}{rgb}{.75,0,0}
\definecolor{lightgrey}{rgb}{.7,.7,.7}

\definecolor{clemson-orange}{RGB}{234,106,32}
\definecolor{chicago-maroon}{RGB}{128,0,0}
\definecolor{northwestern-purple}{RGB}{82,0,99}
\definecolor{cornell-red}{RGB}{179,27,27}
\definecolor{sauder-green}{RGB}{171,180,0}
\definecolor{lawngreen}{RGB}{0,250,154}

\setcounter{MaxMatrixCols}{10}

\onehalfspacing
\newtheorem{theorem}{Theorem}
\newtheorem{acknowledgement}{Acknowledgement}
\newtheorem{algorithm}{Algorithm}
\newtheorem{assumption}{Assumption}
\newtheorem{axiom}{Axiom}
\newtheorem{case}{Case}
\newtheorem{claim}{Claim}
\newtheorem{conclusion}{Conclusion}
\newtheorem{condition}{Condition}
\newtheorem{conjecture}{Conjecture}
\newtheorem{corollary}{Corollary}
\newtheorem{criterion}{Criterion}
\newtheorem{definition}{Definition}
\newtheorem{example}{Example}
\newtheorem{exercise}{Exercise}
\newtheorem{lemma}{Lemma}
\newtheorem{notation}{Notation}
\newtheorem{problem}{Problem}
\newtheorem{proposition}{Proposition}
{\theorembodyfont{\normalfont}
\newtheorem{remark}{Remark}
}
\newtheorem{summary}{Summary}
\newenvironment{proof}[1][Proof]{\textbf{#1.} }{\hfill \rule{0.5em}{0.5em} \bigskip}
\newenvironment{soln}[1][Soln]{\textbf{#1:} }{\hfill \rule{0.5em}{0.5em}}
\renewcommand{\cite}{\citeasnoun}
\renewcommand{\theenumii}{(\alph{enumii})}
\renewcommand{\labelenumii}{\theenumii}
\renewcommand{\theenumiii}{\roman{enumiii}}
\renewcommand{\labelenumiii}{\theenumiii.}

\usepackage[nameinlink]{cleveref}
\crefname{assumption}{Assumption}{Assumptions}
\crefname{lemma}{Lemma}{Lemmas}
\crefname{theorem}{Theorem}{Theorems}
\crefname{corollary}{Corollary}{Corollaries}
\crefname{proposition}{Proposition}{Propositions}
\crefname{claim}{Claim}{Claims}
\crefname{procedure}{Procedure}{Procedures}
\crefname{algorithm}{Algorithm}{Algorithms}
\crefname{figure}{Figure}{Figures}
\crefname{remark}{Remark}{Remarks}
\crefname{section}{Section}{Sections}
\crefname{procedure}{Procedure}{Procedures}
\crefname{example}{Example}{Examples}
\crefname{definition}{Definition}{Definitions}
\crefname{table}{Table}{Tables}
\crefname{align}{}{}
\crefname{enumi}{}{}
\crefname{conjecture}{Conjecture}{Conjectures}
\crefname{step}{Step}{Steps}
\crefname{appendix}{Appendix}{Appendices}
\crefname{footnote}{Footnote}{Footnotes}

\begin{document}


\begin{center}
\textbf{ECON 201 Week 6 Problem Set}
\end{center}

\begin{enumerate}
\item For each of the following statements, determine whether
the statement is \emph{true} or \emph{false} (the latter including ``not necessarily true''), and provide a brief explanation.

    \begin{enumerate}
   	\item A risk-loving decision-maker always takes a risky bet.
	
	\item If two lotteries have the same expected value but different variances, then a risk-averse person always prefers the lottery with a higher variance.

    \item If a risk-averse person is willing to take a risky investment, then a risk-neutral person with the same income is willing to take the same risky investment.

    \item Lane is not risk averse. If he is offered a chance to pay \$10 for a lottery ticket that will give him a prize of \$100 with probability .06, a prize of \$50 with probability .1, and no prize with probability .84, then he will buy the ticket.

    \end{enumerate}

\item Qing just inherited a vineyard from a distant relative. In good years, she earns \$216 from the sale of grapes from the vineyard. If the weather is poor, she earns only \$27. Qing's estimate of the probability of good weather is $2/3$.
	\begin{enumerate}
	\item Compute the expected value and the variance of Qing's income from the vineyard.
	
	\item Qing is risk averse. William, a grape buyer, offers Qing a guaranteed payment of \$153 each year in exchange for her entire harvest. Will Qing accept the offer? Provide a brief explanation.
	
	\item Suppose that Qing's utility function is given by $u(v)=v^{1/3}$. Find the lowest guaranteed payment Qing would accept (in other words, the price at which Qing is indifferent between accepting and rejecting William's offer).
    \end{enumerate}

\item If you remain healthy, you expect to earn an income of \$160,000. If, by contrast, you become disabled, you will only be able to work part time, and your average income will drop to \$40,000. Suppose that you believe that there is a 5 percent chance that you could become disabled. Furthermore, your utility function is $\sqrt{I}$.
    \begin{enumerate}
    \item What is your expected income? What is your expected utility?

    \item Consider an insurance policy that fully insures you in the event that you are disabled. What is the actuarially fair insurance premium? In other words, suppose that you pay $x$ for an insurance that gives you \$120,000 extra income when you become disabled. For what value of $x$ is the insurance actuarially fair?

    \item What is the highest premium that you would be willing to pay for the full insurance policy?
    \end{enumerate}

\end{enumerate}

\end{document}


