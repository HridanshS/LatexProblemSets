\documentclass[11pt]{article}

\usepackage{natbib}
\usepackage{setspace}
\usepackage[left=2.5cm,top=2.8cm,right=2.5cm,bottom=2.8cm]{geometry}
\usepackage{graphicx}
\usepackage{amsmath}
\usepackage{theorem}
\usepackage{version}
\usepackage{multirow}
\usepackage{amssymb}
\usepackage{tikz}
\usetikzlibrary{arrows,arrows.meta,decorations,decorations.pathreplacing,calc,matrix}

\definecolor{Red}{rgb}{1,0,0}
\definecolor{Blue}{rgb}{0,0,1}
\definecolor{Green}{rgb}{0,1,0}
\definecolor{magenta}{rgb}{1,0,.6}
\definecolor{lightblue}{rgb}{0,.5,1}
\definecolor{lightpurple}{rgb}{.6,.4,1}
\definecolor{gold}{rgb}{.6,.5,0}
\definecolor{orange}{rgb}{1,0.4,0}
\definecolor{hotpink}{rgb}{1,0,0.5}
\definecolor{newcolor2}{rgb}{.5,.3,.5}
\definecolor{newcolor}{rgb}{0,.3,1}
\definecolor{newcolor3}{rgb}{1,0,.35}
\definecolor{darkgreen1}{rgb}{0, .35, 0}
\definecolor{darkgreen}{rgb}{0, .6, 0}
\definecolor{darkred}{rgb}{.75,0,0}
\definecolor{lightgrey}{rgb}{.7,.7,.7}

\definecolor{clemson-orange}{RGB}{234,106,32}
\definecolor{chicago-maroon}{RGB}{128,0,0}
\definecolor{northwestern-purple}{RGB}{82,0,99}
\definecolor{cornell-red}{RGB}{179,27,27}
\definecolor{sauder-green}{RGB}{171,180,0}
\definecolor{lawngreen}{RGB}{0,250,154}

\setcounter{MaxMatrixCols}{10}

\onehalfspacing
\newtheorem{theorem}{Theorem}
\newtheorem{acknowledgement}{Acknowledgement}
\newtheorem{algorithm}{Algorithm}
\newtheorem{assumption}{Assumption}
\newtheorem{axiom}{Axiom}
\newtheorem{case}{Case}
\newtheorem{claim}{Claim}
\newtheorem{conclusion}{Conclusion}
\newtheorem{condition}{Condition}
\newtheorem{conjecture}{Conjecture}
\newtheorem{corollary}{Corollary}
\newtheorem{criterion}{Criterion}
\newtheorem{definition}{Definition}
\newtheorem{example}{Example}
\newtheorem{exercise}{Exercise}
\newtheorem{lemma}{Lemma}
\newtheorem{notation}{Notation}
\newtheorem{problem}{Problem}
\newtheorem{proposition}{Proposition}
{\theorembodyfont{\normalfont}
\newtheorem{remark}{Remark}
}
\newtheorem{summary}{Summary}
\newenvironment{proof}[1][Proof]{\textbf{#1.} }{\hfill \rule{0.5em}{0.5em} \bigskip}
\newenvironment{soln}[1][Soln]{\textbf{#1:} }{\hfill \rule{0.5em}{0.5em}}
\renewcommand{\cite}{\citeasnoun}
\renewcommand{\theenumii}{(\alph{enumii})}
\renewcommand{\labelenumii}{\theenumii}
\renewcommand{\theenumiii}{\roman{enumiii}}
\renewcommand{\labelenumiii}{\theenumiii.}

\usepackage[nameinlink]{cleveref}
\crefname{assumption}{Assumption}{Assumptions}
\crefname{lemma}{Lemma}{Lemmas}
\crefname{theorem}{Theorem}{Theorems}
\crefname{corollary}{Corollary}{Corollaries}
\crefname{proposition}{Proposition}{Propositions}
\crefname{claim}{Claim}{Claims}
\crefname{procedure}{Procedure}{Procedures}
\crefname{algorithm}{Algorithm}{Algorithms}
\crefname{figure}{Figure}{Figures}
\crefname{remark}{Remark}{Remarks}
\crefname{section}{Section}{Sections}
\crefname{procedure}{Procedure}{Procedures}
\crefname{example}{Example}{Examples}
\crefname{definition}{Definition}{Definitions}
\crefname{table}{Table}{Tables}
\crefname{align}{}{}
\crefname{enumi}{}{}
\crefname{conjecture}{Conjecture}{Conjectures}
\crefname{step}{Step}{Steps}
\crefname{appendix}{Appendix}{Appendices}
\crefname{footnote}{Footnote}{Footnotes}

\begin{document}


\begin{center}
\textbf{ECON 201 Week 11 Problem Set}\\
\textit {Professor: Teddy kim};  
Sunday, November 13th.
\\Student name: Hridansh Saraogi
\end{center}

\begin{enumerate}
\item For each of the following statements, determine whether
the statement is \emph{true} or \emph{false} (the latter including ``not necessarily true''), and provide a brief explanation.
    \begin{enumerate}
    \item If the amount of a good supplied is independent of the price (i.e., the market supply curve is a vertical line) then a sales tax imposed on the good does not change the price consumers pay.

    \item If there are negative externalities in production or consumption, competitive equilibrium is unlikely to be Pareto efficient but positive externalities enhance the efficiency of the market.
	
	
	\item It is socially desirable to give a subsidy to a firm that generates a positive externality.
    \end{enumerate}

\item Market demand and market supply for mangoes are, respectively, given by
    \begin{equation*}
        D(p)=\frac{91-p}{5}\text{ and }S(p)=\frac{p-3}{6}.
    \end{equation*}
    \begin{enumerate}
        \item Find the competitive equilibrium price $p^{\ast}$ and quantity $q^{\ast}$ for this market.

        \item Suppose the government imposes a tax of \$44 on each unit sold. Find the competitive equilibrium outcome (the price consumers pay and the total quantity), and compute the resulting deadweight loss.

        \item Now, suppose the government gives a subsidy of \$44 on each unit sold to producers. Find the competitive equilibrium outcome, and compute the resulting deadweight loss.
    \end{enumerate}

\item A competitive refining industry produces one unit of waste for each unit of refined product. The industry disposes of the waste by releasing it into the atmosphere. The inverse demand curve for the refined product is given by $p=24-Q$. The private marginal cost curve for refining is given by $MC^{p}=4+Q$, while the marginal external cost (marginal cost of pollution) is given by $MC^{e}=0.5Q$.
	\begin{enumerate}
	\item Find the competitive market outcome (price and quantity) for the refined product when there is no correction for the externality.
	
    \item Find the socially optimal level of output (taking into account the negative externality) and, using the result, calculate the deadweight loss due to the negative externality in (a).


   	\item Suppose that the government imposes an emissions fee of \$t per unit of emissions. How large should the emissions fee be if the market is to produce the socially optimally level of output?
	\end{enumerate}

\end{enumerate}



\end{document}


