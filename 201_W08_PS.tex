\documentclass[11pt]{article}

\usepackage{natbib}
\usepackage{setspace}
\usepackage[left=2.5cm,top=2.8cm,right=2.5cm,bottom=2.8cm]{geometry}
\usepackage{graphicx}
\usepackage{amsmath}
\usepackage{theorem}
\usepackage{version}
\usepackage{multirow}
\usepackage{amssymb}
\usepackage{tikz}
\usetikzlibrary{arrows,arrows.meta,decorations,decorations.pathreplacing,calc,matrix}

\definecolor{Red}{rgb}{1,0,0}
\definecolor{Blue}{rgb}{0,0,1}
\definecolor{Green}{rgb}{0,1,0}
\definecolor{magenta}{rgb}{1,0,.6}
\definecolor{lightblue}{rgb}{0,.5,1}
\definecolor{lightpurple}{rgb}{.6,.4,1}
\definecolor{gold}{rgb}{.6,.5,0}
\definecolor{orange}{rgb}{1,0.4,0}
\definecolor{hotpink}{rgb}{1,0,0.5}
\definecolor{newcolor2}{rgb}{.5,.3,.5}
\definecolor{newcolor}{rgb}{0,.3,1}
\definecolor{newcolor3}{rgb}{1,0,.35}
\definecolor{darkgreen1}{rgb}{0, .35, 0}
\definecolor{darkgreen}{rgb}{0, .6, 0}
\definecolor{darkred}{rgb}{.75,0,0}
\definecolor{lightgrey}{rgb}{.7,.7,.7}

\definecolor{clemson-orange}{RGB}{234,106,32}
\definecolor{chicago-maroon}{RGB}{128,0,0}
\definecolor{northwestern-purple}{RGB}{82,0,99}
\definecolor{cornell-red}{RGB}{179,27,27}
\definecolor{sauder-green}{RGB}{171,180,0}
\definecolor{lawngreen}{RGB}{0,250,154}

\setcounter{MaxMatrixCols}{10}

\onehalfspacing
\newtheorem{theorem}{Theorem}
\newtheorem{acknowledgement}{Acknowledgement}
\newtheorem{algorithm}{Algorithm}
\newtheorem{assumption}{Assumption}
\newtheorem{axiom}{Axiom}
\newtheorem{case}{Case}
\newtheorem{claim}{Claim}
\newtheorem{conclusion}{Conclusion}
\newtheorem{condition}{Condition}
\newtheorem{conjecture}{Conjecture}
\newtheorem{corollary}{Corollary}
\newtheorem{criterion}{Criterion}
\newtheorem{definition}{Definition}
\newtheorem{example}{Example}
\newtheorem{exercise}{Exercise}
\newtheorem{lemma}{Lemma}
\newtheorem{notation}{Notation}
\newtheorem{problem}{Problem}
\newtheorem{proposition}{Proposition}
{\theorembodyfont{\normalfont}
\newtheorem{remark}{Remark}
}
\newtheorem{summary}{Summary}
\newenvironment{proof}[1][Proof]{\textbf{#1.} }{\hfill \rule{0.5em}{0.5em} \bigskip}
\newenvironment{soln}[1][Soln]{\textbf{#1:} }{\hfill \rule{0.5em}{0.5em}}
\renewcommand{\cite}{\citeasnoun}
\renewcommand{\theenumii}{(\alph{enumii})}
\renewcommand{\labelenumii}{\theenumii}
\renewcommand{\theenumiii}{\roman{enumiii}}
\renewcommand{\labelenumiii}{\theenumiii.}

\usepackage[nameinlink]{cleveref}
\crefname{assumption}{Assumption}{Assumptions}
\crefname{lemma}{Lemma}{Lemmas}
\crefname{theorem}{Theorem}{Theorems}
\crefname{corollary}{Corollary}{Corollaries}
\crefname{proposition}{Proposition}{Propositions}
\crefname{claim}{Claim}{Claims}
\crefname{procedure}{Procedure}{Procedures}
\crefname{algorithm}{Algorithm}{Algorithms}
\crefname{figure}{Figure}{Figures}
\crefname{remark}{Remark}{Remarks}
\crefname{section}{Section}{Sections}
\crefname{procedure}{Procedure}{Procedures}
\crefname{example}{Example}{Examples}
\crefname{definition}{Definition}{Definitions}
\crefname{table}{Table}{Tables}
\crefname{align}{}{}
\crefname{enumi}{}{}
\crefname{conjecture}{Conjecture}{Conjectures}
\crefname{step}{Step}{Steps}
\crefname{appendix}{Appendix}{Appendices}
\crefname{footnote}{Footnote}{Footnotes}

\begin{document}


\begin{center}
\textbf{ECON 201 Week 8 Problem Set}\\
\textit {Professor: Teddy kim};  
Sunday, October 23rd.
\\Student name: Hridansh Saraogi
\end{center}

\begin{enumerate}
\item For each of the following statements, determine whether
the statement is \emph{true} or \emph{false} (the latter including ``not necessarily true''), and provide a brief explanation.
    \begin{enumerate}
        \item If a production function $f(x)$ with one input $x$ increases until $20$ and decreases thereafter, then the firm's marginal product is increasing if and only if $x<20$.
        \begin{enumerate}
            \item False
            \item Marginal Product (MP) is the derivative of the production function
            \item When the production function is increasing, the MP is greater than 0 but this does not imply that the MP is increasing
            \item For example, with each additional unit of x, the MP might be decreasing by 10. So on the course from x=0 to x=20, the MP might go from MP=200 to MP=0. Throughout this time, the MP was decreasing yet the total production value (as represented by the production function) was increasing.
        \end{enumerate}

        \item With the same production function in (a), the average product necessarily falls if $x>20$.
        \begin{enumerate}
            \item True
            \item When the value of the production function is decreasing, it signifies that the Marginal Product (MP) is negative. \\
            Each additional unit brings fewer (negative) products to the company. 
            \item Given this, if units (in x) are continued to be increased but the total product is falling, then the average product ($\frac{total product}{No. of units}$) is also bound to fall.
            \item Therefore, if MP is negative, AP will necessarily fall as well
        \end{enumerate}

        \item It a firms' average cost is decreasing at $y$ then its marginal cost is also decreasing at $y$.
        \begin{enumerate}
            \item False
            \item Marginal Cost (MC) is the cost to produce each additional unit
            \item If MC is decreasing (each additional unit is cheaper than the previous unit) then AC is also bound to decrease (the new unit's cost will pull down the previous average cost)
            \item In other words, AC is decreasing when $MC<AC$. When MC=AC, AC stops decreasing. This signifies that MC is decreasing until MC reaches the lowest point of AC.
        \end{enumerate}

        \item If the marginal cost is decreasing until $10$ and then increasing, then the average cost is minimized at $x=10$.
        \begin{enumerate}
            \item False
            \item Average Cost is minimized when AC=MC\\
            When MC is at its lowest point, it does not imply that AC is at its lowest point as well.
            \item This implies that as long as $MC<AC$, AC will continue to decrease\\
            $\therefore$ the average cost is not minimized at $x=10$, but rather when AC=MC
            
        \end{enumerate}

    \end{enumerate}

\item For each of the following situations, find the cost-minimizing input bundle to produce $y=12$ and compute the resulting cost.
    \begin{enumerate}
        \item $f(x_{1},x_{2})=3x_{1}+4x_{2}$, $w_{1}=2$, $w_{2}=3$.
        \begin{enumerate}
            \item MRTS$_{12} = \frac{MP_1}{MP_2} = \frac{4}{3}$\\
            $\frac{w_1}{w_2}=\frac{2}{3}$
            \item Since $\frac{4}{3} > \frac{2}{3}$, we need to increase $x_1$ in order to have $3x_1 = 12$
            \item $\therefore x_1 = 4, x_2 = 0$. This gives us:\\
            $c = w_1 * x_1 + w_2*x_2 = 2(4) = 8$
            \item Hence, the cost-minimizing input bundle is (4,0) with cost 8
        \end{enumerate}

        \item $f(x_{1},x_{2})=\min\{3x_{1},4x_{2}\}$, $w_{1}=2$, $w_{2}=3$.
        \begin{enumerate}
            \item $3x_1 = 4x_2 = 12$\\
            $x_1 = 4$ and $x_2 = 3$
            \item $c = w_1 * x_1 + w_2*x_2 = 2(4) + 3(3)$\\
            c = 17
            \item Hence, the cost-minimizing input bundle is (4,3) with cost 17
        \end{enumerate}

        \item $f(x_{1},x_{2})=x_{1}^{1/2}x_{2}^{1/2}$, $w_{1}=1$, $w_{2}=2$.
        \begin{enumerate}
            \item $\sqrt{x_1}*\sqrt{x_2} = 12$
            \item MRTS$_{12} = \frac{MP_1}{MP_2} = \frac{\frac{1}{2}*\frac{1}{\sqrt{x_1}}*\sqrt{x_2}}{\frac{1}{2}*\frac{1}{\sqrt{x_2}}*\sqrt{x_1}} = \frac{x_2}{x_1} = \frac{1}{2}$
            \item $2x_2 = x_1$\\
            $x_1 = 12\sqrt{2}$ and  $x_2 = 6\sqrt{2}$
            \item Hence, the cost-minimizing input bundle is $(12\sqrt{2},6\sqrt{2})$ with cost $24\sqrt{2}$
        \end{enumerate}

        \item $f(x_{1},x_{2})=\sqrt{x_{1}}+\sqrt{x_{2}}$, $w_{1}=2$, $w_{2}=1$.
        \begin{enumerate}
            \item $\sqrt{x_1}*\sqrt{x_2} = 12$
            \item MRTS$_{12} = \frac{MP_1}{MP_2} = \frac{\frac{1}{2}*\frac{1}{\sqrt{x_1}}}{\frac{1}{2}*\frac{1}{\sqrt{x_2}}} = 2$
            \item $x_2 = 4x_1$\\
            $x_1 = 16$ and  $x_2 = 64$
            \item $c = w_1 * x_1 + w_2*x_2 = 16(2) + 64(1)$\\
            c = 96
            \item Hence, the cost-minimizing input bundle is $(16, 64)$ with cost 96
            
        \end{enumerate}
    \end{enumerate}

\item Suppose $f(x_{1},x_{2})=x_{1}^{2/3}x_{2}^{2/3}$, and $w_{1}=w_{2}=2$.
    \begin{enumerate}
        \item Suppose that in the short run, $x_{2}$ is fixed at $27$. In this case, the firm's production function effectively reduces to $g(x_{1})=f(x_{1},27)=9x_{1}^{2/3}$. Does this short-run production function exhibit constant, decreasing, or increasing returns to scale? Does your answer suggest the firm's short-run average variable cost to increase, decrease, or stay constant?
        \begin{enumerate}
            \item This short-run production function exhibits decreasing returns to scale.
            \item The firm's short-run average variable cost should be increased.
        \end{enumerate}

        \item Derive the firm's short-run \emph{variable} cost function. Does the firm's (short-run) average variable cost increase, decrease, or stay constant?
        \begin{enumerate}
            \item When $x_1 = 0$: $f(x_1,27) = 0$\\
            When $x_1 > 0$ (is positive): $f(x_1,27) = 9(x_1^{\frac{2}{3}})$
            \item y = $9(x_1^{\frac{2}{3}})$\\
            $x_1 = (\frac{y}{9})^{\frac{3}{2}}$
            \item $c = 2*(\frac{y}{9})^{\frac{3}{2}}+54$
            \item The above cost function can be split into two parts: the variable cost, and the fixed cost. \\
            Variable Cost: $2*(\frac{y}{9})^{\frac{3}{2}}$\\
            Fixed Cost: 54
            \item $\therefore$ Average Variable Cost (AVC) = $\frac{vc}{y}=\frac{2*(\frac{y}{9})^{\frac{3}{2}}}{y} = \frac{2*(\frac{y^{\frac{3}{2}}}{9^{\frac{3}{2}}})}{y}$\\
            $= \frac{2*(\frac{y^{\frac{3}{2}}}{3^{3}})}{y} = \frac{2*(y^{\frac{1}{2}})}{3^{3}} = \frac{2\sqrt{y}}{27}$
            \item Hence we conclude that the Average Variable Cost increases
        \end{enumerate}

        \item In the long run (where $x_{2}$ is no longer fixed), does the firm's production function exhibit constant, decreasing, or increasing returns to scale? Does your answer suggest the firm's (long run) average cost to increase, decrease, or stay constant?
        \begin{enumerate}
            \item The firm's production function exhibits increasing returns to scale
            \item The firm's (long run) average cost will decrease.
        \end{enumerate}

        \item Derive the firm's long-run cost function (as a function of $w_{1}$, $w_{2}$, and $y$). Does the firm's (long-run) average cost increase, decrease, or stay constant?
        \begin{enumerate}
            \item $y = x_1^{\frac{2}{3}}*x_2^{\frac{2}{3}}$
            \item MRTS$_{12} = \frac{MP_1}{MP_2} =$\\
            $\frac{\frac{2}{3}x_1^{\frac{-1}{3}}x_2^{\frac{2}{3}}}{\frac{2}{3}x_1^{\frac{2}{3}}x_2^{\frac{-1}{3}}} = $
            $\frac{\frac{2}{3}x_2^{\frac{1}{3}}x_2^{\frac{2}{3}}}{\frac{2}{3}x_1^{\frac{2}{3}}x_1^{\frac{1}{3}}} = $
            $\frac{x_2^{\frac{1}{3}+{\frac{2}{3}}}}{x_1^{\frac{2}{3}+{\frac{1}{3}}}} = $\\ \\
            $\frac{x_2}{x_1}=1$
            \item $x_1 = x_2$; $y = x_1^{\frac{4}{3}}$
            \item $y^3 = x_1^4$\\
            $y^{\frac{4}{3}} = x_1$
            \item $x_1 = x_2 = y^{\frac{4}{3}}$
            \item cost = $c = 4y^{\frac{3}{4}}$\\
            $AVC = \frac{c}{7} = 4y^{\frac{-1}{4}}$
            \item Hence we can conclude that the (long-run) average variable cost of the firm is decreasing
        \end{enumerate}
    \end{enumerate}

\end{enumerate}

\end{document}


