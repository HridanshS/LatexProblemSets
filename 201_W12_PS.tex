%\documentclass[11pt]{article}
%\usepackage{sw20jart}
%\input{tcilatex}


\documentclass[11pt]{article}

\usepackage{natbib}
\usepackage{setspace}
\usepackage[left=2.5cm,top=2.8cm,right=2.5cm,bottom=2.8cm]{geometry}
\usepackage{graphicx}
\usepackage{amsmath}
\usepackage{theorem}
\usepackage{version}
\usepackage{multirow}
\usepackage{amssymb}
\usepackage{tikz}
\usetikzlibrary{arrows,arrows.meta,decorations,decorations.pathreplacing,calc,matrix}

\definecolor{Red}{rgb}{1,0,0}
\definecolor{Blue}{rgb}{0,0,1}
\definecolor{Green}{rgb}{0,1,0}
\definecolor{magenta}{rgb}{1,0,.6}
\definecolor{lightblue}{rgb}{0,.5,1}
\definecolor{lightpurple}{rgb}{.6,.4,1}
\definecolor{gold}{rgb}{.6,.5,0}
\definecolor{orange}{rgb}{1,0.4,0}
\definecolor{hotpink}{rgb}{1,0,0.5}
\definecolor{newcolor2}{rgb}{.5,.3,.5}
\definecolor{newcolor}{rgb}{0,.3,1}
\definecolor{newcolor3}{rgb}{1,0,.35}
\definecolor{darkgreen1}{rgb}{0, .35, 0}
\definecolor{darkgreen}{rgb}{0, .6, 0}
\definecolor{darkred}{rgb}{.75,0,0}
\definecolor{lightgrey}{rgb}{.7,.7,.7}

\definecolor{clemson-orange}{RGB}{234,106,32}
\definecolor{chicago-maroon}{RGB}{128,0,0}
\definecolor{northwestern-purple}{RGB}{82,0,99}
\definecolor{cornell-red}{RGB}{179,27,27}
\definecolor{sauder-green}{RGB}{171,180,0}
%\definecolor{gray}{RGB}{192,192,192}
\definecolor{lawngreen}{RGB}{0,250,154}

\setcounter{MaxMatrixCols}{10}

\onehalfspacing
\newtheorem{theorem}{Theorem}
\newtheorem{acknowledgement}{Acknowledgement}
\newtheorem{algorithm}{Algorithm}
\newtheorem{assumption}{Assumption}
\newtheorem{axiom}{Axiom}
\newtheorem{case}{Case}
\newtheorem{claim}{Claim}
\newtheorem{conclusion}{Conclusion}
\newtheorem{condition}{Condition}
\newtheorem{conjecture}{Conjecture}
\newtheorem{corollary}{Corollary}
\newtheorem{criterion}{Criterion}
\newtheorem{definition}{Definition}
\newtheorem{example}{Example}
\newtheorem{exercise}{Exercise}
\newtheorem{lemma}{Lemma}
\newtheorem{notation}{Notation}
\newtheorem{problem}{Problem}
\newtheorem{proposition}{Proposition}
{\theorembodyfont{\normalfont}
\newtheorem{remark}{Remark}
}
\newtheorem{summary}{Summary}
\newenvironment{proof}[1][Proof]{\textbf{#1.} }{\hfill \rule{0.5em}{0.5em} \bigskip}
\newenvironment{soln}[1][Soln]{\textbf{#1:} }{\hfill \rule{0.5em}{0.5em}}
\renewcommand{\cite}{\citeasnoun}
\renewcommand{\theenumii}{(\alph{enumii})}
\renewcommand{\labelenumii}{\theenumii}
\renewcommand{\theenumiii}{\roman{enumiii}}
\renewcommand{\labelenumiii}{\theenumiii.}

\usepackage[nameinlink]{cleveref}
\crefname{assumption}{Assumption}{Assumptions}
\crefname{lemma}{Lemma}{Lemmas}
\crefname{theorem}{Theorem}{Theorems}
\crefname{corollary}{Corollary}{Corollaries}
\crefname{proposition}{Proposition}{Propositions}
\crefname{claim}{Claim}{Claims}
\crefname{procedure}{Procedure}{Procedures}
\crefname{algorithm}{Algorithm}{Algorithms}
\crefname{figure}{Figure}{Figures}
\crefname{remark}{Remark}{Remarks}
\crefname{section}{Section}{Sections}
\crefname{procedure}{Procedure}{Procedures}
\crefname{example}{Example}{Examples}
\crefname{definition}{Definition}{Definitions}
\crefname{table}{Table}{Tables}
\crefname{align}{}{}
\crefname{enumi}{}{}
\crefname{conjecture}{Conjecture}{Conjectures}
\crefname{step}{Step}{Steps}
\crefname{appendix}{Appendix}{Appendices}
\crefname{footnote}{Footnote}{Footnotes}

\begin{document}


\begin{center}
\textbf{ECON 201 Week 12 Problem Set}\\
\textit {Professor: Teddy kim};  
Sunday, November 19th.
\\Student name: Hridansh Saraogi
\end{center}

\begin{enumerate}
\item For each of the following statements, determine whether
the statement is \emph{true} or \emph{false} (the latter including ``not necessarily true''), and provide a brief explanation.
    \begin{enumerate}
    \item Perfect price discrimination is efficient and, therefore, preferred by everyone to no price discrimination (i.e., uniform pricing).
    \begin{enumerate}
        \item False\\
        Perfect price discrimination is not efficient
        \item It is only preferred by the producers/suppliers because their surplus (producer surplus) is maximised\\Since the Consumer Surplus is 0, it is not preferred by everyone
    \end{enumerate}
	
	\item The more profit a monopoly makes, the more inefficient is the market outcome.
	\begin{enumerate}
        \item False\\
        In this case, efficiency can be determined based on the size of the Deadweight Loss
        \item To be able to maximise profit, a firm should use perfect price discrimination. This will have no Deadweight Loss, and hence the market would be efficient.
    \end{enumerate}
	

	\item A profit-maximizing monopoly sets its price so that $p=MC(q)$.
	\begin{enumerate}
        \item False\\
        A profit-maximizing monopoly sets its price so that $MR=MC(q)$
    \end{enumerate}
	
	
	\item If a firm's demand becomes more elastic, then the firm's optimal price increases.
	\begin{enumerate}
        \item False\\
        Elasticity is determined by the value of $\frac{P}{MC}$
        \item The lower the value of $\frac{P}{MC}$, the higher the elasticity\\
        $\therefore$ If a firm's demand becomes more elastic, then the price has decreased (or MC has become higher)
    \end{enumerate}
	

	\item If a firm's profit-maximizing price is $100$, while its marginal cost is $60$, then its Lerner index is equal to $0.6$.
	\begin{enumerate}
        \item False\\
        The Lerner Index will be calculated as: $\frac{Price-difference}{Price}$
        \item For this situation, the Lerner index will be $\frac{100-60}{100}=\frac{40}{100}=0.4$
    \end{enumerate}
	

    \item When there are multiple markets, a monopoly firm always prefers price-discriminating them to charging the same price.
    \begin{enumerate}
        \item True\\
        Price discrimination gives the firm the opportunity to make more profit than in the scenario where they maintained the same price across all markets.
        \item $\therefore$ When there are multiple markets, a monopoly firm always prefers price-discriminating them
    \end{enumerate}
	

    \end{enumerate}

\pagebreak
\item Suppose that a monopoly has the demand function $D(p)=30-\frac{1}{2}p$ and the cost function $c(q)=q^{2}$.
	\begin{enumerate}
    \item Find the firm's profit optimal monopoly price and quantity.
    \begin{enumerate}
        \item $P=60-2q$\\
        $MC = 2q$\\
        $MR = 60-4q$
        \item $60-4q = 2q$\\
        $60 = 6q$
        \item $\therefore q = 10$\\
        $p = 60-2(10) = 40$
    \end{enumerate}
	
	
    \item Compute the deadweight loss due to monopoly in this market.
    \begin{enumerate}
        \item Deadweight Loss = $\frac{20(5)}{2} = 10(5) = 50$
    \end{enumerate}
	
	
	\item Find this firm's optimal two-part tariff strategy, $(p,e)$, and compute the resulting profit.
	\begin{enumerate}
        \item  Since P = MC:\\
        $60-2q = 2q$\\
        $60 = 4q$\\
        $q = 15$
        \item p = 2(q) = 2(15) = 30
        \item Two-part tariff strategy: $\frac{15(60-30)}{2} = 225$
        \item Profit: $2(225) = 450$
    \end{enumerate}
	
	
	\end{enumerate}

\pagebreak
\item Consider a monopoly firm that faces the demand function $D_{1}(p)=40-2p$ and the constant marginal cost $MC=10$.
    \begin{enumerate}
    \item Find the optimal monopoly quantity and price.
    \begin{enumerate}
        \item Price = $20 - \frac{q}{2}$\\
        $MR = 20 - q$
        \item MC = MR\\
        $10 = 20 - q$\\
        $q = 20-10$
        \item $\therefore q = 10$\\
        $P = 20 - \frac{q}{2}$\\
        $P = 20 - \frac{10}{2}$\\
        $P = 20 - 5$
        \item $\therefore Price = 15$\\
        Optimal Price is 15 and Optimal Quantity is 10
    \end{enumerate}
	

	\item Compute the deadweight loss due to monopoly in this market.
	\begin{enumerate}
        \item Deadweight Loss = $\frac{5(10)}{2} = \frac{50}{2} = 25$
    \end{enumerate}
	

	\item Now suppose that this firm serves another market in which the demand function is given by $D_{2}(p)=30-p$. If this firm can charge a different price in this market, then what is the profit-maximizing price?
	\begin{enumerate}
        \item $P = 30 - q$\\
        $MR = 30-2q$
        \item $10 = 30 - 2q$\\
        $2q = 20$\\
        $\therefore q = 10$
        \item $P = 30 - 10$\\
        $P = 20$
        \item The Profit maximising price is 20
    \end{enumerate}
	

	\item If the firm is required to charge the same price in the two markets (i.e., no price discrimination), what price should the firm charge?
	\begin{enumerate}
        \item Total Demand = Demand in Market 1 + Demand in Market 2\\
        $D = D_1 + D_2$
        \item Total Demand = $(40 - 2p) + (30 - p)$\\
        $D = 70 - 3p$
        \item $P = \frac{70}{3} - \frac{q}{3}$
        \item $\frac{dp}{dq} = \frac{d(30-q)}{dq} = -1$\\
        $MC = p(q) + p'(q)q$\\
        $MC = 30-q-q$\\
        $MC = 30-2q$
        \item $MC = MR$\\
        $\frac{30}{3} - \frac{70}{3} = \frac{-40}{3}$\\
        $\frac{70}{3} - \frac{2q}{3} = 10$ $-> \frac{-2q}{3} = \frac{30}{3} - \frac{70}{3}$\\
        $\frac{-2q}{3} = \frac{-40}{3}$\\
        $-2q = -40$
        \item $\therefore q = 20$
        \item $P = \frac{70}{3} - \frac{q}{3}$\\
        $P = \frac{70}{3} - \frac{20}{3}$\\
        $\therefore P = \frac{50}{3}$
    \end{enumerate}
	
	
	\item What is the value of price discrimination to this firm? In other words, calculate how much more profit this firm can collect in the two markets when it can charge different prices than when it is required to charge the same price.
	\begin{enumerate}
        \item Profit = $\frac{20}{3}(20)$\\
        Profit = $\frac{400}{3}$
        \item Profit$_{type-1} = 10(15-10) = 10(5) = 50$\\
        Profit$_{type-2} = 10(20-10) = 10(10) = 100$\\
        Total Profit = 50 + 100 = 150
        \item Value of price discrimination to the firm = increase in profit by discriminating\\
        Value = $150 - \frac{400}{3} = \frac{450}{3} - \frac{400}{3}$\\
        Value = $\frac{50}{3}$
    \end{enumerate}
	
	
    \end{enumerate}
\end{enumerate}

\end{document}


