\documentclass[11pt]{article}

\usepackage{natbib}
\usepackage{setspace}
\usepackage[left=2.5cm,top=2.8cm,right=2.5cm,bottom=2.8cm]{geometry}
\usepackage{graphicx}
\usepackage{amsmath}
\usepackage{theorem}
\usepackage{version}
\usepackage{multirow}
\usepackage{amssymb}
\usepackage{tikz}
\usetikzlibrary{arrows,arrows.meta,decorations,decorations.pathreplacing,calc,matrix}

\definecolor{Red}{rgb}{1,0,0}
\definecolor{Blue}{rgb}{0,0,1}
\definecolor{Green}{rgb}{0,1,0}
\definecolor{magenta}{rgb}{1,0,.6}
\definecolor{lightblue}{rgb}{0,.5,1}
\definecolor{lightpurple}{rgb}{.6,.4,1}
\definecolor{gold}{rgb}{.6,.5,0}
\definecolor{orange}{rgb}{1,0.4,0}
\definecolor{hotpink}{rgb}{1,0,0.5}
\definecolor{newcolor2}{rgb}{.5,.3,.5}
\definecolor{newcolor}{rgb}{0,.3,1}
\definecolor{newcolor3}{rgb}{1,0,.35}
\definecolor{darkgreen1}{rgb}{0, .35, 0}
\definecolor{darkgreen}{rgb}{0, .6, 0}
\definecolor{darkred}{rgb}{.75,0,0}
\definecolor{lightgrey}{rgb}{.7,.7,.7}

\definecolor{clemson-orange}{RGB}{234,106,32}
\definecolor{chicago-maroon}{RGB}{128,0,0}
\definecolor{northwestern-purple}{RGB}{82,0,99}
\definecolor{cornell-red}{RGB}{179,27,27}
\definecolor{sauder-green}{RGB}{171,180,0}
\definecolor{lawngreen}{RGB}{0,250,154}

\setcounter{MaxMatrixCols}{10}

\onehalfspacing
\newtheorem{theorem}{Theorem}
\newtheorem{acknowledgement}{Acknowledgement}
\newtheorem{algorithm}{Algorithm}
\newtheorem{assumption}{Assumption}
\newtheorem{axiom}{Axiom}
\newtheorem{case}{Case}
\newtheorem{claim}{Claim}
\newtheorem{conclusion}{Conclusion}
\newtheorem{condition}{Condition}
\newtheorem{conjecture}{Conjecture}
\newtheorem{corollary}{Corollary}
\newtheorem{criterion}{Criterion}
\newtheorem{definition}{Definition}
\newtheorem{example}{Example}
\newtheorem{exercise}{Exercise}
\newtheorem{lemma}{Lemma}
\newtheorem{notation}{Notation}
\newtheorem{problem}{Problem}
\newtheorem{proposition}{Proposition}
{\theorembodyfont{\normalfont}
\newtheorem{remark}{Remark}
}
\newtheorem{summary}{Summary}
\newenvironment{proof}[1][Proof]{\textbf{#1.} }{\hfill \rule{0.5em}{0.5em} \bigskip}
\newenvironment{soln}[1][Soln]{\textbf{#1:} }{\hfill \rule{0.5em}{0.5em}}
\renewcommand{\cite}{\citeasnoun}
\renewcommand{\theenumii}{(\alph{enumii})}
\renewcommand{\labelenumii}{\theenumii}
\renewcommand{\theenumiii}{\roman{enumiii}}
\renewcommand{\labelenumiii}{\theenumiii.}

\usepackage[nameinlink]{cleveref}
\crefname{assumption}{Assumption}{Assumptions}
\crefname{lemma}{Lemma}{Lemmas}
\crefname{theorem}{Theorem}{Theorems}
\crefname{corollary}{Corollary}{Corollaries}
\crefname{proposition}{Proposition}{Propositions}
\crefname{claim}{Claim}{Claims}
\crefname{procedure}{Procedure}{Procedures}
\crefname{algorithm}{Algorithm}{Algorithms}
\crefname{figure}{Figure}{Figures}
\crefname{remark}{Remark}{Remarks}
\crefname{section}{Section}{Sections}
\crefname{procedure}{Procedure}{Procedures}
\crefname{example}{Example}{Examples}
\crefname{definition}{Definition}{Definitions}
\crefname{table}{Table}{Tables}
\crefname{align}{}{}
\crefname{enumi}{}{}
\crefname{conjecture}{Conjecture}{Conjectures}
\crefname{step}{Step}{Steps}
\crefname{appendix}{Appendix}{Appendices}
\crefname{footnote}{Footnote}{Footnotes}

\begin{document}


\begin{center}
\textbf{ECON 201 Week 15 Problem Set}\\
Due on \underline{Saturday, December 10th.}
\\
\textit {Professor: Teddy kim};  
Saturday, December 10th.
\\Student name: Hridansh Saraogi

\end{center}

\begin{enumerate}
\item For each of the following statements, determine whether
the statement is \emph{true} or \emph{false} (the latter including ``not necessarily true''), and provide a brief explanation.
    \begin{enumerate}
    \item In a duopoly (2-firm oligopoly), if you can choose to be either a Cournot competitor or a Stackelberg leader, you will always choose to be a Stackelberg leader.
    \begin{enumerate}
        \item True
        \item Firm 1 chooses the Cournot quantity, which leads to Firm 2 to choose the Cournot quantity as well\\
        But Firm 1 can produce more, inducing Firm 2 to choose less. 
        \item Therefore, Firm 1 can have higher profits, and will always choose to be a Stackelber leader
    \end{enumerate}
	
	\item Two firms in an oligopoly can always do better if one firm buys the other or they form a cartel.
    \begin{enumerate}
        \item True
        \item With Collusion (or merging), they only need to care about themselves and restrain their production.
        \item In this case, they can produce less and induce a higher profit
    \end{enumerate}
	
	\item Bertrand price competitors can recover some market power and make positive profits when they differentiate their products.
	\begin{enumerate}
        \item True
        \item With production differentiation, they can produce above Marginal Cost and hence recover their market power
    \end{enumerate}
    
	\item If there are only two firms in a market, then the market outcome cannot be efficient.
    \begin{enumerate}
        \item False
        \item In Bertrand Equilibrium:\\
        $P_1 = P_2 = MC$\\
        Hence the market outcome is efficient (when it is identical products)
    \end{enumerate}
    
	\end{enumerate}
    
    \pagebreak
\item There are two firms selling an identical product, one with constant marginal cost $MC_{1}=2$ and the other with constant marginal cost $MC_{2}=4$. The inverse demand curve for the industry is given by $p=36-2(q_{1}+q_{2})$.
    \begin{enumerate}
        \item Derive firm 1's best response $r_{1}(q_{2})$ to firm 2's quantity $q_{2}$.
        \begin{enumerate}
            \item Given $q_2$ is fixed:\\
            $P = 3b - 2q_1 - 2q_2$\\
             $P = (3b - 2q_2) - 2q_1$
            \item $p (q_1) -(q_1)$\\
            $ = 36q_1 - 2q_1^2 - 2q_2q_1-2q_1$
            \item $q_1 = \frac{34-2q_2}{4}$
            \item MC = MR\\
            $MR = 36 - 2q_2 - 4q_1 = 2$
            \item $q_1 = \frac{34-2q_2}{4}$
            
        \end{enumerate}

        \item Derive firm 2's best response $r_{2}(q_{1})$ to firm 1's quantity $q_{1}$.
        \begin{enumerate}
            \item $q_1$ is fixed
            \item $P = (36-2q_1) - 2q_2$\\
            $MR = 36 - 2q_1 -4q_2 = 4$
            \item $q_2 = \frac{32-2q_1}{4}$
        \end{enumerate}
        
        \item Find the Cournot (Nash) equilibrium for this market, and compute each firm's equilibrium profit.
        \begin{enumerate}
            \item $q_1 = \frac{34 - \frac{32-2q_1}{2}}{4}$\\
            $q_2 = 5$; $P = 36 - 22 = 14$
            \item $4q_1 = \frac{68-32+2q_1}{2}$
            \item $TV_1 = (6)(14-2) = 72$
            \\$TV_2 = (5)(14-4) = 50$
            \item $8q_1 = 36 + 2q_1$\\
            $q_1 = 6$
        \end{enumerate}
        
        \item Find the Stackelberg equilibrium outcome (each firm's output and the resulting price) when firm 1 is the Stackelberg leader.
        \begin{enumerate}
            \item Given Firm 2's response:\\
            $q_2 = \frac{32-2q_1}{4}$
            \item $max(36q_1 - 2q_1^2 - 2q_2q_1 - 2q_1)$\\
            =$q_1(34-2q_1-2q_2)$\\
            $=q_1(34-\frac{32-2q_1}{2}-2q_1)$
            \\$=q_1(\frac{36+2q_1}{2}-2q_1)$\\
            =$q_1(18-q_1) = 18q_1 - q_1^2$
            \item $F.O.C = 18 - 2q_1 = 0$\\
            $q_1 = 9$\\ $q_2 = 3.5$\\ $p = 36 - 25 = 11$
            \item $TV_1 = (11-2)(9) = 81$\\
            $TV_2 = (\frac{7}{2})(11-4) = \frac{49}{2}$
        \end{enumerate}
        
        \item Find the Stackelberg equilibrium outcome (each firm's output and the resulting price) when firm 2 is the Stackelberg leader.
        \begin{enumerate}
            \item Given Firm 1's response:\\
            $q_1 = \frac{34-2q_2}{4}$
            \item $max(36q_1 - 2q_1 - 2q_2)q_2 - 4q_1$\\
            $=q_2(32-2q_1-2q_2)=(32 - \frac{34-2q_2}{2}-2q_2)q_2$\\
            $=q_2(\frac{64-34+2q_2}{2})$\\
            $=q_2(15-q_2)$
            \item $F.O.C = 15q_2 - q_2^2  \hspace{0.5cm}  \rightarrow  \hspace{0.5cm}   15-2q_2 = 0$
            \item $q_2 = 7.5$\\ $P = 11.5$\\ $q_1 = 4.75$
            \item $TV_1 = (4.75)(11.5-2) = 45.125$\\
            $TV_2 = (7.5)(11.5-4) = 56.25$
        \end{enumerate}
    \end{enumerate}
\pagebreak
\item Suppose there are $3$ firms, each with cost function $c(q)=q$ (so each firm's marginal cost is always $1$), and the inverse demand curve is given by $p=13-(q_{1}+q_{2}+q_{3})$.
    \begin{enumerate}
        \item Find firm 1's best response $r_{1}(q_{2},q_{3})$ to the other firms' quantities, $(q_{2},q_{3})$.
        \begin{enumerate}
            \item $P(q_2, q_3) = 13-q_2-q_3-q_1$\\
            $q_1 = \frac{12-q_2-q_3}{2}$
        \end{enumerate}

        \item Find the Cournot equilibrium of this industry. (You can use the fact that all firms are symmetric and so will produce the same quantity under Cournot competition.)
        \begin{enumerate}
            \item $q_1 = q_2 - q_3$\\
            P = 4
        \end{enumerate}

        \item Now suppose firm 1 chooses its quantity $q_{1}$ first, and the other firms choose their quantity after observing $q_{1}$.
            \begin{enumerate}
                \item Given $q_{1}$, firms 2 and 3 engage in standard Cournot competition. Find the Cournot equilibrium of that \emph{subgame}. It suffices to consider $q_{1}<12$.
                \begin{enumerate}
                    \item $P(q_2) = (13-q_1 -q_3) - q_2$\\
                    $MR = (13 - q_1 - q_3) - 2q_2$\\
                    $MR = MC = 1$
                    \item $q_2 = \frac{12 - q_1 - q_2}{4} = 4 -  \frac{q_1}{3}$\\
                    $q_3 = \frac{12-q_2-q_1}{4} = 4 - \frac{q_1}{3}$
                    \item Obtain: $(q_1, 4-\frac{q_1}{3}, 4-\frac{q_1}{3})$
                \end{enumerate}

                \item Using the result in i (which can be represented by $q_{2}(q_{1})$ and $q_{3}(q_{1})$), find firm 1's optimal quantity, $q_{1}^{\ast}$. This problem is effectively identical to that of Stackelberg equilibrium. The only difference is that now there are two followers, not just one.
                \begin{enumerate}
                    \item $P = 13 - q_1 - q_2 - q_3 = 5 - \frac{q_1}{3}$\\
                    $MR = 5 - \frac{2}{3}q_1 = 1$
                    \item Hence: $q_1 = 6$; $q_2 = q_3 = 2$\\
                    quantity = 6
                \end{enumerate}
                
                \item Determine the equilibrium price, and compute each firm's equilibrium profit.
                \begin{enumerate}
                    \item $TV_1 = 6(3-1) = 12$
                    \item $TV_2 = 2(3-1) = 4$
                    \item $TV_3 = 2(3-1) = 4$
                \end{enumerate}
                
                \end{enumerate}
    \end{enumerate}
\pagebreak
\item Suppose that two firms, firm 1 and firm 2, engage in Bertrand competition for differentiated products, each with zero marginal cost. The demand curve for each firm is given as follows:
	\begin{equation*}
	D_{1}(p_{1},p_{2})=30-2p_{1}+p_{2}\text{ and }D_{2}(p_{1},p_{2})=50-3p_{2}+p_{1}.
	\end{equation*}
	\begin{enumerate}
	\item Derive firm 1's best-response function $p_{1}=r_{1}(p_{2})$ to firm 2's price $p_{2}$.
    \begin{enumerate}
        \item $(p_{1}D_{1})_{max} = p1(30-2p_1+p_2)$ 
        \\$= 30p_1 - 2p_1^2+p_1p_2$
        \item F.O.C:\\
        $30-4p_1 + p_2 = 0$
        \item $p_1 = \frac{30 + p_2}{4}$
    \end{enumerate}

    \item Derive firm 2's best-response function $p_{2}=r_{2}(p_{1})$ to firm 1's price $p_{1}$.
    \begin{enumerate}
        \item $p_2 (50 - 3p_2 + p_1)$\\
        $= 50p_2 - 3p_2^2 + p_1p_2$
        \item F.O.C:\\
        $50 - 6p_2 + p_1 = 0$\\
        $p_2 = \frac{50 + p_1}{6}$
    \end{enumerate}
	\item Find the Bertrand equilibrium in this market and calculate each firm's profit.
    \begin{enumerate}
        \item $p_1 = p_2$\\
        $\frac{30+p_2}{4} = p_2$
        \item $p_2 = 10, p_1 = 10$\\
        $D_1(10,10) = 20$; $D_2(10,10) = 30$\\
        $P = 10$; $D = 50$
        \item $TV_1 = 200$; $TV_2 = 300$
    \end{enumerate}

	\end{enumerate}

\end{enumerate}

\end{document}


