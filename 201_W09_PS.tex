\documentclass[11pt]{article}

\usepackage{natbib}
\usepackage{setspace}
\usepackage[left=2.5cm,top=2.8cm,right=2.5cm,bottom=2.8cm]{geometry}
\usepackage{graphicx}
\usepackage{amsmath}
\usepackage{theorem}
\usepackage{version}
\usepackage{multirow}
\usepackage{amssymb}
\usepackage{tikz}
\usetikzlibrary{arrows,arrows.meta,decorations,decorations.pathreplacing,calc,matrix}

\definecolor{Red}{rgb}{1,0,0}
\definecolor{Blue}{rgb}{0,0,1}
\definecolor{Green}{rgb}{0,1,0}
\definecolor{magenta}{rgb}{1,0,.6}
\definecolor{lightblue}{rgb}{0,.5,1}
\definecolor{lightpurple}{rgb}{.6,.4,1}
\definecolor{gold}{rgb}{.6,.5,0}
\definecolor{orange}{rgb}{1,0.4,0}
\definecolor{hotpink}{rgb}{1,0,0.5}
\definecolor{newcolor2}{rgb}{.5,.3,.5}
\definecolor{newcolor}{rgb}{0,.3,1}
\definecolor{newcolor3}{rgb}{1,0,.35}
\definecolor{darkgreen1}{rgb}{0, .35, 0}
\definecolor{darkgreen}{rgb}{0, .6, 0}
\definecolor{darkred}{rgb}{.75,0,0}
\definecolor{lightgrey}{rgb}{.7,.7,.7}

\definecolor{clemson-orange}{RGB}{234,106,32}
\definecolor{chicago-maroon}{RGB}{128,0,0}
\definecolor{northwestern-purple}{RGB}{82,0,99}
\definecolor{cornell-red}{RGB}{179,27,27}
\definecolor{sauder-green}{RGB}{171,180,0}
\definecolor{lawngreen}{RGB}{0,250,154}

\setcounter{MaxMatrixCols}{10}

\onehalfspacing
\newtheorem{theorem}{Theorem}
\newtheorem{acknowledgement}{Acknowledgement}
\newtheorem{algorithm}{Algorithm}
\newtheorem{assumption}{Assumption}
\newtheorem{axiom}{Axiom}
\newtheorem{case}{Case}
\newtheorem{claim}{Claim}
\newtheorem{conclusion}{Conclusion}
\newtheorem{condition}{Condition}
\newtheorem{conjecture}{Conjecture}
\newtheorem{corollary}{Corollary}
\newtheorem{criterion}{Criterion}
\newtheorem{definition}{Definition}
\newtheorem{example}{Example}
\newtheorem{exercise}{Exercise}
\newtheorem{lemma}{Lemma}
\newtheorem{notation}{Notation}
\newtheorem{problem}{Problem}
\newtheorem{proposition}{Proposition}
{\theorembodyfont{\normalfont}
\newtheorem{remark}{Remark}
}
\newtheorem{summary}{Summary}
\newenvironment{proof}[1][Proof]{\textbf{#1.} }{\hfill \rule{0.5em}{0.5em} \bigskip}
\newenvironment{soln}[1][Soln]{\textbf{#1:} }{\hfill \rule{0.5em}{0.5em}}
\renewcommand{\cite}{\citeasnoun}
\renewcommand{\theenumii}{(\alph{enumii})}
\renewcommand{\labelenumii}{\theenumii}
\renewcommand{\theenumiii}{\roman{enumiii}}
\renewcommand{\labelenumiii}{\theenumiii.}

\usepackage[nameinlink]{cleveref}
\crefname{assumption}{Assumption}{Assumptions}
\crefname{lemma}{Lemma}{Lemmas}
\crefname{theorem}{Theorem}{Theorems}
\crefname{corollary}{Corollary}{Corollaries}
\crefname{proposition}{Proposition}{Propositions}
\crefname{claim}{Claim}{Claims}
\crefname{procedure}{Procedure}{Procedures}
\crefname{algorithm}{Algorithm}{Algorithms}
\crefname{figure}{Figure}{Figures}
\crefname{remark}{Remark}{Remarks}
\crefname{section}{Section}{Sections}
\crefname{procedure}{Procedure}{Procedures}
\crefname{example}{Example}{Examples}
\crefname{definition}{Definition}{Definitions}
\crefname{table}{Table}{Tables}
\crefname{align}{}{}
\crefname{enumi}{}{}
\crefname{conjecture}{Conjecture}{Conjectures}
\crefname{step}{Step}{Steps}
\crefname{appendix}{Appendix}{Appendices}
\crefname{footnote}{Footnote}{Footnotes}

\begin{document}


\begin{center}
\textbf{ECON 201 Week 9 Problem Set}\\
\textit {Professor: Teddy kim};  
Sunday, October 30th.
\\Student name: Hridansh Saraogi
\end{center}


\begin{enumerate}
\item For each of the following statements, determine whether
the statement is \emph{true} or \emph{false} (the latter including ``not necessarily true''), and provide a brief explanation.
    \begin{enumerate}

    \item Cost minimization is not a necessary condition for profit maximization; that is, even if a firm fails to minimize its production cost, it can still maximize its profit.
    \begin{enumerate}
        \item False
        \item Cost minimization is a necessary condition for profit maximization.
        \item All firms are attempting to maximize profits. In general, there is only one level of output that satisfies that goal. When the firm is producing at the profit maximizing output level, it chooses the combination of inputs that minimizes the cost of producing that output level.
        \item If you are producing at the profit maximization output without minimizing the production cost, Your profit is less.
        \item Profit = Revenue - Cost\\The lower the cost, the more profit you have.
    \end{enumerate}

    \item Suppose a firm's production function with one factor $x$ is given by $f(x)=305x-2x^{2}$. If the price of its output is $p=2$ and the price of its factor is $w=10$ then the firm should use $75$ units of $x$.
    \begin{enumerate}
        \item True
        \item $MRP = p * MP = w$\\
        $2(305-4x)=10$
        \item x = 75
        
    \end{enumerate}

    \item In the short run, a firm may produce a positive quantity even if its overall profit is negative.
    \begin{enumerate}
        \item True
        \item $If P > AVC$ but $P < ATC$:\\
        the firm continues to produce in the short-run, making economic losses
        \item In the short run, a firm will continue to produce if it can earn enough revenues to cover at least its variable costs. 
        \item In this case, a firm may be producing at a price above AVC. Despite that, the firm would continue to produce in the short run even if it is making a negative profit, but it would exit the market eventually in the long run (if the overall profit continues to be negative).
    \end{enumerate}

    \item For a market to be perfectly competitive, different firms must sell sufficiently differentiated products.
    \begin{enumerate}
        \item False
        \item Firms should be selling identical products for thr market to be called perfectly competitive
    \end{enumerate}

    \item The cheese business in Lake Fon-du-lac, Wisconsin, is a competitive industry. All cheese manufacturers have the cost function $c(q)=q^{2}+16$, while demand for cheese in the town is given by $D(p)=120-p$. The long-run equilibrium number of firms in this industry will be 28.
    \begin{enumerate}
        \item True
        \item Cost: $c(q)=q^{2}+16$\\
        Demand: $D(p)=120-p$
        \item MC(q) = AC(q)\\
        $2q = q+\frac{16}{q}$\\
        $q = \frac{16}{q}$
        \item $q^2 = 16$\\
        $q=4$\\
        $q* = 4$
        \item $p* = MC(4) = AC(4)$\\
        $p* = 2q = 2(4) = 8$
        \item $D(p*) = 112 = S(p*) = n*(q)$\\
        112 = n*(4)
        \item $\therefore n* = 28$
    \end{enumerate}
    \end{enumerate}

\pagebreak
\item Suppose $f(x_{1},x_{2})=\min\{\sqrt{x_{1}},\sqrt{x_{2}}\}$, $w_{1}=2$, and $w_{2}=3$.
    \begin{enumerate}
        \item Does this function exhibit constant, decreasing, or increasing returns to scale?
        \begin{enumerate}
            \item Decreasing Returns to Scale
            \item $min\{\sqrt{1}, \sqrt{1}\} = 1$\\
            $2(min\{\sqrt{1}, \sqrt{1}\}) = 2$
            \item $min\{\sqrt{2}, \sqrt{2}\} = 1.41$
            \item $\therefore$ we determine that we are witnessing Decreasing Returns to Scale
        \end{enumerate}

        \item In the short run, $x_{2}$ is fixed at $100$. Find this firm's short-run cost function $sc(y)$ and output supply function $y(p)$. Note that this firm cannot produce more than $10$ (because $f(x_{1},100)=\min\{\sqrt{x_{1}},\sqrt{100}\}\leq 10$), so effectively $c(y)=\infty$ if $y>10$.
        \begin{enumerate}
            \item $w_1 = 2; w_2 = 3$
            \item $x_2 = 100$; find sc(y) and y(p)\\
            $f(x_1, 100) = min\{\sqrt{x_1}, 10\}$\\
            $y \leq 10$
            \item $f(x_1, 100) = min\{\sqrt{x_1}, 10\}$\\
            $f(x_1, 100)$ = y = $\sqrt{x_1}$\\
            $y^2 = x_1 \rightarrow x_1 = y^2$
            \item $sc(y) = 2y^2 + 300$\\
            $y(p) = py - sc(y)$
            \item $p = mc(y) = 4y$\\
            $y = \frac{p}{4}$
            
        \end{enumerate}

        \item Find this firm's long-run cost function $lc(y)$ and output supply function $y(p)$.
        \begin{enumerate}
            \item $f(x_1, x_2) = min\{\sqrt{x_1}, \sqrt{x_2}\}$
            \item 1) $y = \sqrt{x_1} = \sqrt{x_2}$\\
            2) $\sqrt{x_1} = \sqrt{x_2}$\\
            $y = \sqrt{x_1} \rightarrow y^2 = x_1$
            \item $lc(y) = 2y^2 + 3y^2 = 5y^2$
            \item $mc() = p = 10y$\\
            $10y = p \rightarrow y = \frac{p}{10}$
        \end{enumerate}

    \end{enumerate}
    
\pagebreak
\item Consider the following perfectly competitive market: each firm's long-run cost function is given by $C(q)=72+8q+\frac{q^{2}}{2}$ (for $q>0$), and market demand is given by $D(p)=400-2p$.
    \begin{enumerate}
    \item How much would each individual firm produce in the long-run competitive equilibrium?
    \begin{enumerate}
        \item 
    \end{enumerate}

    \item Find the long-run competitive equilibrium price in this industry.
    \begin{enumerate}
        \item 
    \end{enumerate}

    \item How many firms will operate in this industry in the long run?
    \begin{enumerate}
        \item 
    \end{enumerate}

    \end{enumerate}

\end{enumerate}
\end{document}




