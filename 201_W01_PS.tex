\documentclass[11pt]{article}

\usepackage{natbib}
\usepackage{setspace}
\usepackage[left=2.5cm,top=2.8cm,right=2.5cm,bottom=2.8cm]{geometry}
\usepackage{graphicx}
\usepackage{amsmath}
\usepackage{theorem}
\usepackage{version}
\usepackage{multirow}
\usepackage{amssymb}
\usepackage{tikz}
\usetikzlibrary{arrows,arrows.meta,decorations,decorations.pathreplacing,calc,matrix}

\definecolor{Red}{rgb}{1,0,0}
\definecolor{Blue}{rgb}{0,0,1}
\definecolor{Green}{rgb}{0,1,0}
\definecolor{magenta}{rgb}{1,0,.6}
\definecolor{lightblue}{rgb}{0,.5,1}
\definecolor{lightpurple}{rgb}{.6,.4,1}
\definecolor{gold}{rgb}{.6,.5,0}
\definecolor{orange}{rgb}{1,0.4,0}
\definecolor{hotpink}{rgb}{1,0,0.5}
\definecolor{newcolor2}{rgb}{.5,.3,.5}
\definecolor{newcolor}{rgb}{0,.3,1}
\definecolor{newcolor3}{rgb}{1,0,.35}
\definecolor{darkgreen1}{rgb}{0, .35, 0}
\definecolor{darkgreen}{rgb}{0, .6, 0}
\definecolor{darkred}{rgb}{.75,0,0}
\definecolor{lightgrey}{rgb}{.7,.7,.7}

\definecolor{clemson-orange}{RGB}{234,106,32}
\definecolor{chicago-maroon}{RGB}{128,0,0}
\definecolor{northwestern-purple}{RGB}{82,0,99}
\definecolor{cornell-red}{RGB}{179,27,27}
\definecolor{sauder-green}{RGB}{171,180,0}
%\definecolor{gray}{RGB}{192,192,192}
\definecolor{lawngreen}{RGB}{0,250,154}

\setcounter{MaxMatrixCols}{10}

\onehalfspacing
\newtheorem{theorem}{Theorem}
\newtheorem{acknowledgement}{Acknowledgement}
\newtheorem{algorithm}{Algorithm}
\newtheorem{assumption}{Assumption}
\newtheorem{axiom}{Axiom}
\newtheorem{case}{Case}
\newtheorem{claim}{Claim}
\newtheorem{conclusion}{Conclusion}
\newtheorem{condition}{Condition}
\newtheorem{conjecture}{Conjecture}
\newtheorem{corollary}{Corollary}
\newtheorem{criterion}{Criterion}
\newtheorem{definition}{Definition}
\newtheorem{example}{Example}
\newtheorem{exercise}{Exercise}
\newtheorem{lemma}{Lemma}
\newtheorem{notation}{Notation}
\newtheorem{problem}{Problem}
\newtheorem{proposition}{Proposition}
{\theorembodyfont{\normalfont}
\newtheorem{remark}{Remark}
}
\newtheorem{sumAlice}{SumAlice}
\newenvironment{proof}[1][Proof]{\textbf{#1.} }{\hfill \rule{0.5em}{0.5em} \bigskip}
\newenvironment{soln}[1][Soln]{\textbf{#1:} }{\hfill \rule{0.5em}{0.5em}}
\renewcommand{\cite}{\citeasnoun}
\renewcommand{\theenumii}{(\alph{enumii})}
\renewcommand{\labelenumii}{\theenumii}
\renewcommand{\theenumiii}{\roman{enumiii}}
\renewcommand{\labelenumiii}{\theenumiii.}


\usepackage[nameinlink]{cleveref}
\crefname{assumption}{Assumption}{Assumptions}
\crefname{lemma}{Lemma}{Lemmas}
\crefname{theorem}{Theorem}{Theorems}
\crefname{corollary}{Corollary}{Corollaries}
\crefname{proposition}{Proposition}{Propositions}
\crefname{claim}{Claim}{Claims}
\crefname{procedure}{Procedure}{Procedures}
\crefname{algorithm}{Algorithm}{Algorithms}
\crefname{figure}{Figure}{Figures}
\crefname{remark}{Remark}{Remarks}
\crefname{section}{Section}{Sections}
\crefname{procedure}{Procedure}{Procedures}
\crefname{example}{Example}{Examples}
\crefname{definition}{Definition}{Definitions}
\crefname{table}{Table}{Tables}
\crefname{align}{}{}
\crefname{enumi}{}{}
\crefname{conjecture}{Conjecture}{Conjectures}
\crefname{step}{Step}{Steps}
\crefname{appendix}{Appendix}{Appendices}
\crefname{footnote}{Footnote}{Footnotes}

\begin{document}


\begin{center}
\textbf{ECON 201 Week 1 Problem Set}\\
\textit {Professor: Teddy kim};  
Sunday, September 4th.
\\Group 5: Hridansh Saraogi, ZhenYan Li, Luis Gardner, Pedro Choo

\end{center}

\begin{enumerate}
\item Windred consumes only apples ($a$) and bananas ($b$). He prefers more apples to fewer, but he gets tired of bananas. If he consumes fewer than 24 bananas per week, he thinks that 1 banana is a perfect substitute for 1 apple. But you would have to pay him 1 apple for each banana beyond 24 that he consumes. For example, Windred is indifferent among $(1,23)$, $(0,24)$, and $(1,25)$. In each of the following problems, determine $x$ that makes Windred indifferent between the two given consumption bundle and provide a brief explanation.
    \begin{enumerate}
        \item $(13,23)\sim (x,0)$
        \begin{enumerate}
            \item Ans: The value of x should be 23+13 which is 36. This is because the number of bananas (b) is less than 24. As per the problem statement, If Windred consumes fewer than 24 bananas per week, he thinks that 1 banana is a perfect substitute for 1 apple. Given this, let's focus on the 'b' component of the problem: $(13,23)$. Suppose the consumption bundle was $(0,23)$, Windred would be equally satisfied with $(23,0)$. So $(0,23)\sim (23,0)$.
            Now focusing on the 'a' component of the original problem, 13 apples remains unaffected. Given $(0,23)\sim (23,0)$, we can add the original 13 apples to $(23,0)$, making it 23+13 apples = 36 apples. Therefore, the value of x is 36. 
        \end{enumerate}
        \item $(40,40)\sim (x,0)$
        \begin{enumerate}
            \item Ans: The value of x should be 48
            $(40,40)\sim (48,0)$
            # of 'b' exceeds 24 by 16, creating a negative utility of -16
            \\16 'a' are required to pay for the additional 16 'b' Windred consumed. Therefore, the total utility is $(40-16)$ + $(40-16)$ = 48
            \\Given that $(x, 0)$, X should be 48 to satisfy the total utility, which was 48.
        \end{enumerate}
        \item $(31,36)\sim (x,18)$
        \begin{enumerate}
            \item Ans: x is 25
            \\Let's begin by calculating the total utility in $(31,36)$.
            \\Since you have to pay Windred 1 apple for each banana beyond 24 that he consumes, any bananas consumed beyond 24 is technically negative utility for him. Since in $(31,36)$, 36 is beyond 24, Windred is actually given 12 bananas which have negative utility. For these 12 bananas, Windred must be paid with 12 extra Apples.
            \\So in the 31 apples, he was actually given 19 apples, and 12 'a's to compensate for the negative utility of the extra 'b'
            \\So his total utility was in fact 19 + 24 = 43
            \\Since Windred continues to keep 18 'b', as given in $(x,18)$, he needs to be given sufficient 'a' to maintain his total utility. 43 - 18 = 25
            \\Therefore, Windred should have 25 'a' (hence x = 25)
        \end{enumerate}
    \end{enumerate}
    
\pagebreak

\item Alice Granola consumes only avocados $(x_{1})$ and grapefruits $(x_{2})$ and never gets tired of them (i.e., she always prefers to have more of each good). If we graph Alice Granola’s indifference curves with avocados on the horizontal axis and grapefruits on the vertical axis, then whenever she has more grapefruits than avocados, the (absolute value of) slope of her indifference curve is 2. Whenever she has more avocados than grapefruits, the (absolute value of) slope is 1/3. Determine whether each of the following statements is true or false, and provide a brief explanation.
    \begin{enumerate}
        \item Alice would prefer $(9,12)$ to $(12,9)$.
        \begin{enumerate}
            \item Ans: TRUE
            \\Reasoning: Alice would prefer $(9,12)$ to $(12,9)$. This can be calculated by adding the number of avocados. We start from the consumption bundle of $(9,12)$.
            \begin{itemize}
                \item $(10,10)$ +1 avocado, -2 grapefruit
                \item $(11, 9.66)$ +1 avocado, -1/3 grapefruit
                \item $(12, 9.33)$ +1 avocado, -1/3 grapefruit
                \item $(13,9)$ +1 avocado, -1/3 grapefruit
            \end{itemize}
            \\Point $(9,12)$ and $(13,9)$ are on the same indifference curve. However, $(12,9)$ is not. It is below $(13,9)$ which is below the indifference curve. 
            \\
            \textit{Thus, Alice would prefer $(9,12)$ over $(12,9)$}
        \end{enumerate}
        \item Alice would be indifferent between $(14,20)$ and $(22,14)$.
        \begin{enumerate}
            \item Ans: TRUE
            \\Reasoning: Using the same method utilized for $(a)$, we can start with $(14,20)$
            \begin{itemize}
                \item $(15,18)$ +1 avocado, -2 grapefruit
                \item $(16,16)$ +1 avocado, -2 grapefruit
                \item $(19,15)$ -1 grapefruit,+3 avocado
                \item $(22,14)$ -1 grapefruit,+3 avocado
            \end{itemize}
            \\We can see that (14, 20) and (22, 14) are on the same indifference curve. \\
            \textit{Thus, Alice is indifferent between these consumption bundles.}
        \end{enumerate}
        \item Alice's preferences are convex.
        \begin{enumerate}
            \item Ans: TRUE
            \\Reasoning: Alice’s preferences are convex because she always prefers more of each good.\\
            Moreover, if we drew the indifference curve, we could see that any two points on the curve could form a straight line. \\
            Upon finding the average point Z on the line, we observe that it is above the indifference curve. \\
            Thus, we want to consume at point Z (because more is better, according to Alice).\\
            The definition of convex says that 'averages are better than extremes'. Since Point Z is indeed better than the two endpoints (or extremes) of the line which lie on the indifference curve, we would prefer to consume at Z.\\
            \textit{Thus, Alice’s preferences are convex.}
        \end{enumerate}
    \end{enumerate}
    
\pagebreak

\item Suppose Jenny lives for two periods, today (young) and tomorrow (old). Let $c_{1}$ denote her consumption today and $c_{2}$ denote her consumption tomorrow. Also, let $m_{1}$ denote her income today and $m_{2}$ denote her income tomorrow.
    \begin{enumerate}
        \item Suppose that the interest rate for \emph{savings} is given by $r_{S}$ (i.e., if she saves \$x today then she will get back \$$(1+r_{S})x$ tomorrow). Compute the maximum amount Jenny can consume tomorrow.
        \begin{itemize}
            \item $c_{1}$ = 0 \hspace{1cm} \Rightarrow \hspace{1cm} $c_{2}$ = $(1+r_{S})$ $m_{1}$ + $m_{2}$.
        \end{itemize}

        \item Suppose that the interest rate for \emph{borrowing} is given by $r_{B}$ (i.e., if she borrows \$x today then she has to pay back \$$(1+r_{B})x$ tomorrow). Compute the maximum amount Jenny can consume today.
        \begin{itemize}
            \item $c_{2}$ = 0 \hspace{1cm} \Rightarrow \hspace{1cm} $c_{1}$ = $m_{1}$ + ($m_{2}$)/$(1 + r_{B})$
        \end{itemize}

        \item Suppose $r_{S}\leq r_{B}$ (as is typically the case). In this case, Jenny's budget (or choice) set can be represented by two mathematical inequalities. Find those two inequalities. Note that the budget line crosses the points in (a) and (b) as well as the endowment $(m_{1},m_{2})$. (It will help if you draw the budget line to understand the structure of the budget set.)
        \begin{itemize}
            \item Tomorrow’s perspective:\\
            $(1 + r_{s})$ * $c_{1}$ + 1*$c_{2}$ \hspace{1cm} \leq \hspace{1cm} $(1 + r_{s})$ * $m_{1}$ + $m_{2}$
            \item Today’s perspective: \\
            1 * $c_{1}$ + 1/(1+$r_{B}$)*$c_{2}$ \hspace{0.8cm} \leq \hspace{1cm} $m_{1}$ + $m_{2}$/(1 + $r_{B}$)
        \end{itemize}
    \end{enumerate}


\end{enumerate}

\end{document}

