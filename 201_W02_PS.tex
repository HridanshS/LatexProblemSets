\documentclass[11pt]{article}

\usepackage{natbib}
\usepackage{setspace}
\usepackage[left=2.5cm,top=2.8cm,right=2.5cm,bottom=2.8cm]{geometry}
\usepackage{graphicx}
\usepackage{amsmath}
\usepackage{theorem}
\usepackage{version}
\usepackage{multirow}
\usepackage{amssymb}
\usepackage{tikz}
\usetikzlibrary{arrows,arrows.meta,decorations,decorations.pathreplacing,calc,matrix}

\definecolor{Red}{rgb}{1,0,0}
\definecolor{Blue}{rgb}{0,0,1}
\definecolor{Green}{rgb}{0,1,0}
\definecolor{magenta}{rgb}{1,0,.6}
\definecolor{lightblue}{rgb}{0,.5,1}
\definecolor{lightpurple}{rgb}{.6,.4,1}
\definecolor{gold}{rgb}{.6,.5,0}
\definecolor{orange}{rgb}{1,0.4,0}
\definecolor{hotpink}{rgb}{1,0,0.5}
\definecolor{newcolor2}{rgb}{.5,.3,.5}
\definecolor{newcolor}{rgb}{0,.3,1}
\definecolor{newcolor3}{rgb}{1,0,.35}
\definecolor{darkgreen1}{rgb}{0, .35, 0}
\definecolor{darkgreen}{rgb}{0, .6, 0}
\definecolor{darkred}{rgb}{.75,0,0}
\definecolor{lightgrey}{rgb}{.7,.7,.7}

\definecolor{clemson-orange}{RGB}{234,106,32}
\definecolor{chicago-maroon}{RGB}{128,0,0}
\definecolor{northwestern-purple}{RGB}{82,0,99}
\definecolor{cornell-red}{RGB}{179,27,27}
\definecolor{sauder-green}{RGB}{171,180,0}
%\definecolor{gray}{RGB}{192,192,192}
\definecolor{lawngreen}{RGB}{0,250,154}

\setcounter{MaxMatrixCols}{10}

\onehalfspacing
\newtheorem{theorem}{Theorem}
\newtheorem{acknowledgement}{Acknowledgement}
\newtheorem{algorithm}{Algorithm}
\newtheorem{assumption}{Assumption}
\newtheorem{axiom}{Axiom}
\newtheorem{case}{Case}
\newtheorem{claim}{Claim}
\newtheorem{conclusion}{Conclusion}
\newtheorem{condition}{Condition}
\newtheorem{conjecture}{Conjecture}
\newtheorem{corollary}{Corollary}
\newtheorem{criterion}{Criterion}
\newtheorem{definition}{Definition}
\newtheorem{example}{Example}
\newtheorem{exercise}{Exercise}
\newtheorem{lemma}{Lemma}
\newtheorem{notation}{Notation}
\newtheorem{problem}{Problem}
\newtheorem{proposition}{Proposition}
{\theorembodyfont{\normalfont}
\newtheorem{remark}{Remark}
}
\newtheorem{summary}{Summary}
\newenvironment{proof}[1][Proof]{\textbf{#1.} }{\hfill \rule{0.5em}{0.5em} \bigskip}
\newenvironment{soln}[1][Soln]{\textbf{#1:} }{\hfill \rule{0.5em}{0.5em}}
\renewcommand{\cite}{\citeasnoun}
\renewcommand{\theenumii}{(\alph{enumii})}
\renewcommand{\labelenumii}{\theenumii}
\renewcommand{\theenumiii}{\roman{enumiii}}
\renewcommand{\labelenumiii}{\theenumiii.}

\usepackage[nameinlink]{cleveref}
\crefname{assumption}{Assumption}{Assumptions}
\crefname{lemma}{Lemma}{Lemmas}
\crefname{theorem}{Theorem}{Theorems}
\crefname{corollary}{Corollary}{Corollaries}
\crefname{proposition}{Proposition}{Propositions}
\crefname{claim}{Claim}{Claims}
\crefname{procedure}{Procedure}{Procedures}
\crefname{algorithm}{Algorithm}{Algorithms}
\crefname{figure}{Figure}{Figures}
\crefname{remark}{Remark}{Remarks}
\crefname{section}{Section}{Sections}
\crefname{procedure}{Procedure}{Procedures}
\crefname{example}{Example}{Examples}
\crefname{definition}{Definition}{Definitions}
\crefname{table}{Table}{Tables}
\crefname{align}{}{}
\crefname{enumi}{}{}
\crefname{conjecture}{Conjecture}{Conjectures}
\crefname{step}{Step}{Steps}
\crefname{appendix}{Appendix}{Appendices}
\crefname{footnote}{Footnote}{Footnotes}

\begin{document}


\begin{center}
\textbf{ECON 201 Week 2 Problem Set}\\
\textit {Professor: Teddy kim};  
Sunday, September 11th.
\\Group 5: Hridansh Saraogi, ZhenYan Li, Luis Gardner, Pedro Choo
\end{center}

\begin{enumerate}

\item For each of the following cases, find the consumer's optimal consumption bundle. As usual, $p_{i}$ denotes the price of good $i$, and $m$ denotes the consumer's income (budget).
    \begin{enumerate}
        \item $u(x_{1},x_{2})=\min\{x_{1},x_{2}\}$, $p_{1}=2$, $p_{2}=1$, and $m=24$. In this case, you cannot apply the arbitrage method and should directly find the optimal consumption bundle.
        \begin{enumerate}
            \item (8,8)
            \item This is perfect competition. \\
            Therefore, optimal consumption bundle happens when $x_{1}=x_{2}$.
            \item Budget line: $2x_{1}+x_{2}=24$
            \item Substitute $x_{1}$ into $x_{2}$ we get: $2x_{1}+x_{1}=24 \rightarrow x_{1}=8$\\
            $\therefore x_{2} = 8$ 
            \\ Plug in (8,8) into budget line equation and it equals 24
        \end{enumerate}

        \item $u(x_{1},x_{2})=\min\{2x_{1},3x_{2}\}$, $p_{1}=p_{2}=1$, and $m=20$.
        \begin{enumerate}
            \item (12,8)
            \item Perfect competition\\
            $\therefore$ Optimal consumption bundle happens when $2x_{1} = 3x_{2}$\\
            Modify utility function and get $x_{1} = \frac{3}{2} x_{2}$
            \item Budget line: $X_{1}+X_{2}=20$\\
            $\frac{3}{2} x_{2} + x_{2} = 20$\\
            $\frac{5}{2}x_{2} = 20$ $\rightarrow$ $x_{2}=8; x_{1}=12$
            \item Plug in (8,8) into budget line equation and it equals 20
        \end{enumerate}

        \item $u(x_{1},x_{2})=3x_{1}+4x_{2}$, $p_{1}=2$, $p_{2}=3$, and $m=36$. You can either directly find the solution or apply the arbitrage method.
        \begin{enumerate}
            \item (18,0)
            \item Budget line: $2X_{1}+3X_{2}=36$
            \item Find MRS by partial differentiation and get $MU_{1}/MU_{2} = \frac{3}{4}$
            \item set MRS equal to slope of budget line: $\frac{3}{4} = \frac{2}{3}$
            \item Since MRS $>$ slope of budget line
            \\We want more of $x_{1}$
            \item We should set $x_{2}$ to 0 to maximize $x_{1}$:\\ $2x_{1}=36$ $\rightarrow$ $x_{1}=18$ \\
            $x_{2}=0$
        \end{enumerate}


        \item Nick's indifference curves are circles, all of which are centered at $(12,12)$. Of any two indifference circles, he would rather be on the inner one than the outer one (i.e., Nick's utility increases as he gets closer to $(12,12)$). In addition, $p_{1}=1$, $p_{2}=2$, and $m=16$.
        \begin{enumerate}
            \item (8,4)
            \item We should use the circle equation to solve this problem\\
            $(x – x_{1})^2 + (y – y_{1})^2= r^2$
            \item $u(x_{1},x_{2})= (x_{1}-x_{2})^2+(x^2-12)^2$
            \\$MRS_{12}$ = $\frac{2x_1 - 24}{2x_2 - 24}$ − $\frac{x_1 - 12}{x2 - 12}$ = $\frac{1}{2}$
            \item $x_2-12 =2x_1-24$ $\hspace{1cm}\rightarrow \hspace{1cm} x_2 = 2x_1-12$
            \item Budget line = $x_1+2x_2=16$\\
            Substitute $2x_1-12$ into $x_2$
            \item $x_1+2(2x_1-12)=16$ \\
            $x_1+4x_1-24=16$
            \item $5x_1 =40$ $\rightarrow x_1 = 8$\\
            $x_2 =4$
        \end{enumerate}

        \item $u(x_{1},x_{2})=\ln(x_{1})+2\ln(x_{2})$ (Cobb-Douglas), $p_{1}=p_{2}=1$, and $m=36$. For this and subsequent two problems, apply the arbitrage method.
        \begin{enumerate}
            \item (12,4)
            \item Partial Differentiation to obtain:\\
            $MU_1 = \frac{1}{x_1}$ $MU_2 = \frac {2}{x_2}$\\
            $MRS_{12} = \frac{x_2}{2x_1}$
            \item Cross multiply: $2x_1=x_2$
            \item Budget line: $x_1+x_2=36$\\
            Substitute $2x_1$ into $x_2: x_1+ 2x_1 =36$
            \item $x_1 =12; x_2 =24$
        \end{enumerate}


        \item $u(x_{1},x_{2})=10\ln(x_{1})+x_{2}$ (quasi-linear), $p_{1}=p_{2}=1$, and $m=36$.
        \begin{enumerate}
            \item (10,26)
            \item Partial Differentiation to obtain:\\
            $MU_1 = \frac{10}{x_1}$ $MU_2 = 1$\\
            $MRS_{12} = \frac{10}{x_1}$
            \item $MRS_{12} = \frac{p_1}{p_2}= \frac{10}{x_1} = \frac{1}{1}$\\
            \item Budget line: $x_1+x_2=36$\\
            $x_1=10$; $x_2=26$
        \end{enumerate}

        \item $u(x_{1},x_{2})=12\sqrt{x_{1}}+x_{2}$ (quasi-linear), $p_{1}=2$, $p_{2}=1$, and $m=16$.
        \begin{enumerate}
            \item (8,0)
            \item Partial Differentiation to obtain:\\
            $MU_1 = \frac{6}{\sqrt{x_1}}$ $MU_2 = 1$\\
            $MRS_{12} = \frac{6}{\sqrt{x_1}}$
            \item $MRS_{12} = \frac{p_1}{p_2}= \frac{6}{\sqrt{x_1}} = \frac{2}{1}$\\
            \item $\sqrt{2x_1} = 6$ $\hspace{0.5 cm}\rightarrow \hspace{0.5 cm}$ $\hspace{0.5 cm} \sqrt{x_1} = 3 \hspace{0.5 cm}$ $\rightarrow$ $\hspace{0.5 cm} x_1 = 9$
            \item Budget line: $2x_1+x_2=16$
            \item $18 + x_2 =16$ $\rightarrow x_2=-2$
            \item We want more of $x_1$\\
            set $x_2 = 0$\\
            $2x_1=16$ $\hspace{0.5 cm} \rightarrow \hspace{0.5 cm}$ $x_1=8$\\
            $x_2=0$
        \end{enumerate}

        \item $u(x_{1},x_{2})=2\sqrt{x_{1}}+\sqrt{x_{2}}$, $p_{1}=2$, $p_{2}=1$, and $m=18$.
        \begin{enumerate}
            \item (6,6)
            \item $MU_1 = \frac{2}{\sqrt{x_1}}$ $MU_2 = \frac{1}{\sqrt{x_2}}$\\
            $MRS_{12} = \frac{2\sqrt{x_2}}{\sqrt{x_1}}$
            \item $MRS_{12} = \frac{p_1}{p_2}= \frac{2\sqrt{x_2}}{\sqrt{x_1}} = \frac{2}{1}$
            \item $2\sqrt{x_1} = 2\sqrt{x_2}$ \\
            $\sqrt{x_1} = \sqrt{x_2}$ $\hspace{0.5cm} \rightarrow \hspace{0.5cm}$ $x_1 = x_2$
            \item Budget line: $2x_1+x_2=18$\\
            Since $x_1=x_2$, we replace $x_2$ with $x_1$
            \item $2x_1+x_1=18$\\
            $x_1=6$; $x_2=6$
            
        \end{enumerate}
    \end{enumerate}

\item Consider Jenny's intertemporal choice problem from Week 1 Problem Set. Let $c_{1}$ denote her consumption today, $c_{2}$ her consumption tomorrow, $m_{1}$ her income today, and $m_{2}$ her income tomorrow. Her preferences over $(c_{1},c_{2})$ can be represented by the utility function $u(c_{1},c_{2})$. She can save at rate $r_{S}$ and borrow at rate $r_{B}$. In each of the following situations, determine whether she will save, borrow, or do neither, and provide a brief explanation.

\textbf{To solve problem 2 we have to understand how to come up with the budget line.}
\textit{Slope of Budget line:} \textit{$P_1 = 1$ $P_2 = \frac{1}{1+r}$; \hspace{1cm} $\frac{P_1}{P_2}  = $ $\frac{1}{\frac{1}{1+r}} = 1 + r$}

    \begin{enumerate}
        \item $u(c_{1},c_{2})=c_{1}c_{2}$, $m_{1}=100$, $m_{2}=144$, and $r_{S}=r_{B}=0.2$.
        \begin{enumerate}
            \item Borrow
            \item $u(c_1,c_2) =c_1c_2$
            \item $MRS_{{m_1},{m_2}} = \frac{c_2}{c_1} = 1 + r = \frac{144}{100} > 1.2$
            \item For(a), $r_s$ and $r_b$ are the same which is 1.2
            \item Our MRS is 1.44\\
            Since $1.44 > 1.2$
            \\Thus we should borrow
        \end{enumerate}

        \item $u(c_{1},c_{2})=c_{1}c_{2}$, $m_{1}=100$, $m_{2}=120$, $r_{S}=0.1$, and $r_{B}=0.15$.
        \begin{enumerate}
            \item Borrow
            \item $u(c_1,c_2) =c_1c_2$
            \item $MU_1 = c_2$; $MU_2 = c_1$
            \item $MRS_{{m_1},{m_2}} = \frac{c_2}{c_1} = 1 + r = \frac{120}{100} > 1.1 \hspace{1cm} (r_s)$ \\
            $\frac{120}{100} > 1.15 \hspace{1cm} (r_b)$
            \item For(b), $r_s$ and $r_b$ are different so we have to compare MRS to $r_s$ and $r_b$ separately
            \item For $r_s$ our MRS $1.2 > 1.1$\\
            For $r_b$ our MRS $1.2 > 1.15$
            \item Since $MRS_{12} > 1+r_b$\\
            we should borrow
        \end{enumerate}

        \item $u(c_{1},c_{2})=\ln(c_{1})+2\ln(c_{2})$, $m_{1}=100$, $m_{2}=200$, $r_{S}=0.1$, and $r_{B}=0.15$.
        \begin{enumerate}
            \item 
        \end{enumerate}

        \item $u(c_{1},c_{2})=\sqrt{c_{1}}+0.9\sqrt{c_{2}}$, $m_{1}=100$, $m_{2}=100$, $r_{S}=0.1$, and $r_{B}=0.15$.
        \begin{enumerate}
            \item 
        \end{enumerate}
    \end{enumerate}


\end{enumerate}

\end{document}

